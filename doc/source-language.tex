\documentclass[a4paper,11pt,brazil]{article}

\usepackage[left=25mm,top=15mm]{geometry}
\usepackage[utf8]{inputenc}
\usepackage[T1]{fontenc}
\usepackage{babel}
\usepackage{hyperref}
\usepackage{lmodern}
\usepackage{indentfirst}
\usepackage{tabularx}
\usepackage{longtable}
\usepackage{graphics}
\usepackage{float}
\usepackage{syntax}
\usepackage{dirtree}
\usepackage{pygmentex}
\usepackage{tcolorbox}
\usepackage{mystyle}

\setpygmented{boxing method=tcolorbox,lang=ocaml,tabsize=2}
\efboxsetup{hidealllines,backgroundcolor=yellow!30}

% syntax configuration
\renewcommand{\syntleft}{\normalfont\slshape\hspace{0.25em}}
\renewcommand{\syntright}{\hspace{0.25em}}
\renewcommand{\ulitleft}{\normalfont\ttfamily\bfseries\frenchspacing\color{red}\hspace{0.25em}}
\renewcommand{\ulitright}{\hspace{0.25em}}


\renewcommand{\DTcomment}[1]{\hfill \sffamily #1}

\newcommand{\tiger}{\textsl{Tiger}}
\newcommand{\lang}{\textsl{Coffee}}


\begin{document}

\begin{center}
  \textbf{Construção de Compiladores I [BCC328]}
\end{center}
\begin{tcolorbox}[colback=yellow!20]
  \begin{center}
    \textbf{\Large Definição e Implementação da linguagem \lang{}}\\[1em]
  \end{center}
\end{tcolorbox}
\begin{center}
  Departamento de Computação\par
  Universidade Federal de Ouro Preto\par
  Prof. José Romildo Malaquias\par
  \today
\end{center}

\begin{abstract}
  \lang{} é uma pequena linguagem de programação usada para fins
  didáticos na aprendizagem de técnicas de construção de
  compiladores.
  
  A documentação e implementação de \lang{} será realizada de forma
  colaborativa pelos participantes do curso.
\end{abstract}

\tableofcontents

\section{A linguagem \lang{}}

Andrew Appel apresenta em seu livro \textsl{Modern Compiler
  Implementation in ML} uma pequena linguagem de programação para fins
didáticos chamada \tiger{}. Nesta disciplina vamos considerar uma
linguagem semelhante, que chamaremos de \lang{}, baseada em \tiger{},
porém com algumas diferenças sintáticas e semânticas.
\begin{center}
  \includegraphics[width=.4\textwidth]{coffee.jpeg}
\end{center}

\lang{} é uma linguagem de programação bastante simples que será usada
para praticarmos a implementação de um compilador com aplicação das
técnicas discutidas nas aulas.

As construções da linguagem serão detalhadas no decorrer o
curso. Começaremos com uma versão básica, e oportunamente serão
apresentadas versões mais aprimoradas, com novas construções.

\lang{} é uma linguagem imperativa com tipagem estática. Seus tipos de
dados básicos são:
\begin{itemize}
  \item \textbf{\texttt{bool}}: valores lógicos
  \item \textbf{\texttt{int}}: valores inteiros de precisão fixa
  \item \textbf{\texttt{real}}: valores reais (números em ponto
  flutuante de dupla precisão)
  \item \textbf{\texttt{string}}: cadeias de caracteres
\end{itemize}


\section{O projeto}

\subsection{Estrutura do projeto}

O projeto será desenvolvido na linguagem OCaml usando a ferramenta
dune para automatização da compilação. O projeto usa algumas
ferrametnas e bibliotecas externas:
\begin{description}
  \item[ocamllex] Ocamllex é um gerador de analisador léxico. Ele
  produz um analisadores léxicos (em OCaml) a partir de conjuntos de
  expressões regulares com ações semânticas associadas. É distribuído
  junto com o compilador de OCaml.
  
  \item[menhir] Menhir é um gerador de analisador sintático. Ele
  transforma especificações gramaticais de alto nível, decoradas com
  ações semânticas expressas na linguagem de programação OCaml, em
  analisadores sintáticos, também expressos em OCaml.

  \item[dune] Dune é um sistema de compilação para OCaml (e Reason).

  \item[ppx_import] É uma extensão de sintaxe que permite extrair
  tipos ou assinaturas de outros arquivos de interface compilados

  \item[ppx_deriving] É uma extensão de sintaxe que facilita geração
  de código baseada em tipos em OCaml

  \item[ppx_expect] É uma extensão de sintaxe para escrita de testes
  em OCaml

  \item[camomile] Camomile é uma biblioteca unicode para OCaml.  
\end{description}
  
O código é organizado segundo a estrutura de diretórios mostrada na
figura~\ref{fig:dir}.
\begin{figure}[H]
  \begin{tcolorbox}
    \dirtree{%
      .1 \lang{}.
      .2 src\DTcomment{código fonte do projeto}.
      .3 lib\DTcomment{biblioteca com as principais definições usadas para montar o compilador}.
      .3 driver\DTcomment{driver do compilador}.
      .2 doc\DTcomment{Documentação de \lang{}}.
    }
  \end{tcolorbox}
  \caption{Estrutura de diretórios do projeto do compilador}
  \label{fig:dir}
\end{figure}
Existem outros diretórios gerados automaticamente que não são
relevantes nesta discussão e por isto não foram mencionados. Caso
algum ambiente de desenvolvimento integrado (IDE) seja usado,
provavelmente haverá alguns arquivos e diretórios específicos do
ambiente e que também não são mencionados.

Os arquivos \texttt{src/lib/dune} e \texttt{src/driver/dune} contêm as
especificações do projeto esperada pelo dune. Neles são indicadas
informações como nome do projeto, dependências externas e flags
necessários para a compilação da biblioteca e da aplicação.

\subsection{Módulos importantes}

Alguns módulos importantes no projeto são mencionados a seguir. Alguns
já estão prontos, e outros deverão ser criados ou modificados pelo
participante nas atividades do curso.

\begin{itemize}
  \item O módulo \pyginline|Absyn| contém os tipos que representam as
  árvores sintáticas abstratas para as construções da linguagem fonte.

  \item O módulo \pyginline|Lexer| contém as declarações relacionadas
  com o analisador léxico do compilador.

  O analisador léxico (módulo \pyginline|Lexer|) é gerado
  automaticamente pelo \texttt{ocamllex}. A especificação léxica é
  feita no arquivo \texttt{src/lib/lexer.mll} usando expressões
  regulares.

  \item O módulo \pyginline|Parser| contém as declarações relacioandas
  com o analisador sintático do compilador.

  O analisador sintático (módulo \pyginline|Parser|) é gerado
  automaticamente pelo \texttt{menhir}. A especificação sintática é
  feita no arquivo \texttt{src/lib/parser.mly} usando uma gramática
  livre de contexto.
  
  \item O módulo \pyginline|Semantic| contém declarações usadas na
  análise semântica e geração de código do compilador.

  \item O módulo \pyginline|Symbol| contém declarações que implementam
  um tipo usado para representar nomes de identificadores, e
  discutido posteriormente.

  \item O módulo \pyginline|Environment| contém declarações para a
  manipulação dos ambientes (ás vezes também chamados de contexto) de
  compilação. Estes ambientes são representados usando tabelas de
  símbolos.

  \item O módulo \pyginline|Types| contém declarações para a
  representação interna dos tipos da linguagem fonte.

  \item O módulo \pyginline|Error| contém declarações usadas para
  reportar errors detectados pelo compilador durante a compila

  \item O módulo \pyginline|Location| contém declarações usadas para
  representação de localizações de erros no código fonte, importantes
  quando os erros forem reportados.

  \item O módulo \pyginline|Driver| é formado por declarações, incluindo
  a função \pyginline|main|, ponto de entrada para execução do
  compilador.
\end{itemize}

\subsection{Acrescentando uma nova construçao de \lang{} }

Ao acrescentar uma nova construção na implementação da linguagem, procure
seguir os seguintes passos:
\begin{itemize}
  \item Se necessário defina um novo construtor de dados no tipo
  \pyginline|Types.t| para representar algum tipo da linguagem \lang{}
  que ainda não faça parte do projeto.

  \item Se necessário acrecente ao ambiente inicial (no módulo
  \pyginline|Environment.env|) a representação de quaisquer novos
  tipos, variáveis ou funções que façam parte da biblioteca padrão de
  \lang{}.

  \item Defina os novos construtores de dados necessárias para
  representar a árvore abstrata para a construção no tipo
  \pyginline|Absyn.t|.
  \begin{itemize}
    \item definir os campos necessários para as sub-árvores da árvore
    abstrata,
    \item extender a função \pyginline|to_tree| que permite converter
    para uma árvore de strings, útil para visualização gráfica da
    árvore abstrata.
  \end{itemize}

  \item Extenda a função de análise semântica
  \pyginline|Semantic.semantic| para tratar a nova construção.

  \item Declare quaisquer novos símbolos terminais e não-terminais na
  gramática livre de contexto da linguagem que se fizerem necessários
  para as especificações léxica e sintática da construção.

  \item Acrescente as regras de produção para a construção na gramática livre
  de contexto da linguagem, tomando o cuidado de escrever ações semânticas
  adequadas para a construção da árvore abstrata correspondente. Se necessário
  use declarações de precedência de operadores.

  \item Se necessário acrescente regras léxicas que permitam reconhecer os
  novos símbolos terminais na especificação léxica da linguagem.
\end{itemize}

\section{Mensagens de erro}

O projeto contém algumas funções para reportar erros encontrados
durante a compilação. Estas funções fazem parte do módulo
\pyginline|Error| e serão comentadas a seguir.

Em todo compilador é desejável que os erros encontrados sejam
reportados com uma indicação da localização do erro, acompanhada por
uma mensagem explicativa do problema ocorrido. Para tanto torna-se
necessário manter a informação da localização em que cada frase do
programa (cada nó da árvore abstrata construída para representar o
programa) foi encontrada. O módulo \pyginline|Location| contém algumas
definições relacionadas com estas localizações.

Neste projeto as localizações no código fonte são representadas pelo
tipo \pyginline|Location.t|, que leva em consideração as posições no
código fonte onde a frase começou e terminou.

Cada uma destas posições é do tipo \pyginline|Lexing.position|. O
módulo \pyginline|Lexing| faz parte da biblioteca padrão do OCaml e
será extensivamente usado nas implementações dos analisadores léxico e
sintático. O tipo \pyginline|Lexing.position| contém as seguintes
informações:
\begin{itemize}
  \item a indicação da unidade (arquivo fonte) sendo compilada,
  \item o número da linha, e
  \item o número da coluna
\end{itemize}

A função \pyginline|Error.error| e outras similares encontradas no
módulo \pyginline|Error| devem ser usadas para emissão de mensagens de
erro. Esta função recebe como argumentos a localização do erro, a
mensagem de formatação de diagnóstico, e possivelmente argumentos
complementares de acordo com a mensagem.


\section{Aspectos léxicos}
\label{sec:lex}

\subsection{Brancos e comentários}

Ocorrências de \textbf{caracteres brancos} (espaço, tabulação
horizontal, nova linha) e comentários entre os símbolos léxicos são
ignorados.

\textbf{Comentários} são ignorados pelo compilador e podem ser úteis
como anotações sobre o programa para alguém que estiver lendo ou
modificando o programa. Os comentários podem ser:
\begin{itemize}
  \item \textbf{comentários de linha}, que em \lang{} começam com o
  caractere \texttt{\#} e se estendem até o final da linha.

  \item \textbf{comentários de bloco}, que são delimitados pelas
  sequências de caracteres \texttt{\{\#} e \texttt{\#\}} e podem ser
  \textbf{aninhados}.
\end{itemize}

Exemplos de comentários:

\begin{pygmented}[lang=text]
# isto é um comentário de linha
\end{pygmented}

\begin{pygmented}[lang=text]
{# isto é um
   comentário de bloco #}
\end{pygmented}

\begin{pygmented}[lang=text]
{# isto é um
   comentário de bloco {# aninhado #}
   percebeu? #}
\end{pygmented}

Os seguintes comentários estão incorretos, pois não foram terminados:

\begin{pygmented}[lang=text]
{# Este comentário de bloco
não terminou!
\end{pygmented}

\begin{pygmented}[lang=text]
{# Este comentário de bloco {# com aninhamento #}
   também não terminou!
\end{pygmented}


\subsection{Literais}

\subsubsection{Literais inteiros}

Os \textbf{literais inteiros} são formados por uma sequência de um ou
mais dígitos decimais.

São exemplos de literais inteiros:
\begin{pygmented}[lang=text]
2014
872834
0
0932
\end{pygmented}

Não há nenhum literal inteiro negativo.

\subsubsection{Literais reais}

Os \textbf{literais reais} correspondem a números em ponto flutuante
possivelmente em notação científica. São formados por uma parte
inteira, seguida de uma parte decimal e/ou uma parte exponencial.

A parte inteira é formada por uma sequência de um ou mais dígitos
decimais.

A parte decimal é formada pelo caracter \texttt{.}, seguido de uma
sequência de um ou mais dígitos decimais.

A parte exponencial representa uma potência de dez é formada por um
dos caracteres \texttt{e} ou \texttt{E}, seguido opcionalmente dos
caracteres \texttt{+} ou \texttt{-}, seguido de uma sequência de um ou
mais dígitos decimais.

São exemplos de literais reais:
\begin{pygmented}[lang=text]
20.14
0.0872834
123.456e12
5632.003E-15
77e100
\end{pygmented}

Não há nenhum literal real negativo.


\subsubsection{Literais lógicos}

Os \textbf{literais lógicos} são \texttt{true} (verdadeiro) e
\texttt{false} (falso).


\subsubsection{Literais string}

Os \textbf{literais string} são formados por uma sequência de
caracteres delimitada por aspas (\texttt{"}). Na sequência de
caracteres o caracter \texttt{\textbackslash} é especial e inicia uma
sequência de escape. Uma sequência de escape representa um caracter de
acordo com a tabela a seguir.

\begin{center}
  \begin{tabularx}{\linewidth}{|l|X|}\hline
    \textbf{sequência de escape}                      & \textbf{descrição}                                                              \\\hline
    \texttt{\textbackslash \textbackslash}            & \texttt{\textbackslash}                                                         \\\hline
    \texttt{\textbackslash "}                         & \texttt{"}                                                                      \\\hline
    \texttt{\textbackslash t}                         & tabuação horizontal                                                             \\\hline
    \texttt{\textbackslash n}                         & nova linha                                                                      \\\hline
    \texttt{\textbackslash r}                         & retorno de carro                                                                \\\hline
    \texttt{\textbackslash b}                         & retrocesso                                                                      \\\hline
    \texttt{\textbackslash \emph{ddd}}                & caracter de código \emph{ddd}, sendo \emph{ddd} uma sequências de 3 dígitos decimais          \\\hline
  \end{tabularx}
\end{center}

Estas são as únicas sequências de escape válidas.

São exemplos de literais string:
\begin{pygmented}[lang=text]
"Tiger"
"Bom dia, Brasil!"
"B"
"\065 = \"B\""
""
"abc\tDEF\nGHI\\JKL\"mno\065ok"
\end{pygmented}

Não são exemplos de literais string:
\begin{pygmented}[lang=text]
# invalid escape sequence in string literal
"abc\kdef"
"\64 is not ok"

# unclosed string literal
"abc
\end{pygmented}


\subsubsection{Identificadores}

\textbf{Identificadores} são sequências de letras maiúsculas ou
minúsculas, dígitos decimais e sublinhados (\texttt{\_}), começando
com uma letra. Letras maiúsculas e minúsculas são distintas em um
identificador.

Identificadores são usados para nomear entidades usadas em um
programa, como tipos, funções e variáveis.

São exemplos de identificadores:
\begin{pygmented}[lang=text]
peso
idadeAluno
alfa34
primeiro_nome
\end{pygmented}

Não são exemplos de identificadores:
\begin{pygmented}[lang=text]
__peso
idade do aluno
34rua
primeiro-nome
\end{pygmented}


\subsubsection{Operadores, sinais de pontuação  e palavras-chave}

Os seguintes operadores podem ser usados em \lang{}:
\begin{pygmented}[lang=text]
+ - * / % ^
= <> > >= < <=
&& ||
:=
\end{pygmented}

São sinais de pontuação em \lang{}:
\begin{pygmented}[lang=text]
( ) , ; :
\end{pygmented}

As palavras-chave de \lang{} são:
\begin{pygmented}[lang=text]
var
if then else
while do break
let in end
\end{pygmented}

Palavras-chave são reservadas, isto é, não podem ser usadas como
identificadores.

Todas as palavras-chave são escritas com letras minúsculas.


\section{Símbolos}

Linguagens de programação usam \textbf{identificadores} para nomear
entidades da linguagem, como tipos, variáveis, funções, classes,
módulos, etc.

Símbolos léxicos (também chamados de símbolos terminais ou
\emph{tokens}) que são classificados como identificadores tem um valor
semântico (atributo) que é o nome do identificador. A princípio o
valor semântico do identificador pode ser representado por uma cadeia
de caracteres (tipo \pyginline|string| do OCaml). Porém o tipo
\pyginline|string| tem algumas incoveniências para o compilador:
\begin{itemize}
  \item Geralmente o mesmo identificador ocorre várias vezes em um
  programa. Se cada ocorrência for representada por uma string (ou
  seja, por uma sequência de caracteres), o uso de memória poderá ser
  grande.

  \item Normalmente existem dois tipos de ocorrência de
  identificadores em um programa:
  \begin{itemize}
    \item uma declaração do identificador, e
    \item um ou mais usos do identificador já declarado.
  \end{itemize}
  Durante a compilação cada ocorrência de uso de um identificador deve
  ser associada com uma ocorrência de declaração. Para tanto os
  identificadores devem ser comparados para determinar se são iguais
  (isto é, se tem o mesmo nome). O uso de strings é ineficiente, pois
  pode ser necessário comparar todos os caracteres da string para
  determinar se elas são iguais ou não.
\end{itemize}

Por estas razões o compilador utiliza o tipo \pyginline|Symbol.symbol|
para representar os nomes dos identificadores. Basicamente mantém-se
uma tabela \emph{hash} onde os identificadores são colocados à medida
que eles são encontrados. Sempre que o analisador léxico encontrar um
identificador, deve-se verificar se o seu nome já está na tabela. Em
caso afirmativo, usa-se o símbolo correspondente que já se encontra na
tabela. Caso contrário cria-se um novo símbolo, que é adicionado à
tabela associado ao nome encontrado, e é usado pelo analisador léxico
como valor semântico do \emph{token}.

A implementação de \pyginline|Symbol.symbol| associa-se a cada novo
símbolo um número inteiro diferente. A comparação de igualdade de
símbolos se resume a uma comparação (muito eficiente) de inteiros, já
que o mesmo identificador estará sempre sendo representado pelo mesmo
símbolo (associado portanto ao mesmo número inteiro).

A função \pyginline|Symbol.symbol| cria um símbolo a partir de uma
string.


\section{O  analisador léxico}

O módulo \pyginline|Lexer| contém as declarações que implementam o
analisador léxico do compilador. Este módulo será gerado
automaticamente pela ferramenta \texttt{ocamllex}.

A especificação da estrutura léxica da linguagem fonte é feita no
arquivo \texttt{src/lib/lexer.mll} usando expressões
regulares. Consulte a documentação do \texttt{ocamllex} para entender
como fazer a especificação léxica.

Os analisadores léxico e sintático vão se comunicar durante a
compilação, pois os tokens obtidos pelo analisador léxico serão
consumidos pelo analisador sintático. Ou seja, os tokens são os
símbolos terminais da gramática usada pelo gerador de analisador
sintático. Para manter a consistência dos analisadores léxico e
sintático os símbolos terminais (tokens) são declarados na gramática
livre de contexto do \texttt{menhir}, no arquivo
\texttt{src/lib/parser.mly}. Consulte a documentação do
\texttt{menhir} para entender como escrever a gramática livre de
contexto.




\section{Sintaxe}


A sintaxe de todas as construções de \lang{} é apresentada na
gramática livre de contexto que se segue.

  \begin{synshorts}
    \small
    \noindent
    \begin{longtable}{r@{$\;\rightarrow\;$}lr}
      <Program>    & <Exp>                                               & programa                \\[.9em]
      <Exp>        & "litint"                                            & literais                \\
      <Exp>        & "litreal"                                           &                         \\
      <Exp>        & "litbool"                                           &                         \\

      <Exp>        & "litstring"                                         &                         \\[.9em]

      <Exp>        & <Var>                                               & variável                \\[.9em]

      <Exp>        & "-" <Exp>                                           & operações aritméticas   \\
      <Exp>        & <Exp> "+" <Exp>                                     &                         \\
      <Exp>        & <Exp> "-" <Exp>                                     &                         \\
      <Exp>        & <Exp> "*" <Exp>                                     &                         \\
      <Exp>        & <Exp> "/" <Exp>                                     &                         \\
      <Exp>        & <Exp> "\%" <Exp>                                    &                         \\
      <Exp>        & <Exp> "^" <Exp>                                     &                         \\[.9em]

      <Exp>        & <Exp> "=" <Exp>                                     & operações relacionais   \\
      <Exp>        & <Exp> "<>" <Exp>                                    &                         \\
      <Exp>        & <Exp> ">" <Exp>                                     &                         \\
      <Exp>        & <Exp> ">=" <Exp>                                    &                         \\
      <Exp>        & <Exp> "<" <Exp>                                     &                         \\
      <Exp>        & <Exp> "<=" <Exp>                                    &                         \\[.9em]

      <Exp>        & <Exp> "\&\&" <Exp>                                  & operações lógicas       \\
      <Exp>        & <Exp> "||" <Exp>                                    &                         \\[.9em]

      <Exp>        & <Var> ":=" <Exp>                                    & atribuição              \\[.9em]

      <Exp>        & "id" "(" <Exps> ")"                                 & chamada de função       \\[.9em]

      <Exp>        & "if" <Exp> "then" <Exp> "else" <Exp>                & expressões condicionais \\
      <Exp>        & "if" <Exp> "then" <Exp>                             &                         \\[.9em]

      <Exp>        & "while" <Exp> "do" <Exp>                            & expressão de repetição  \\
      <Exp>        & "break"                                             &                         \\[.9em]

      <Exp>        & "let" <Decs> "in" <Exp>                             & expressão de declaração \\[.9em]

      <Exp>        & "(" <ExpSeq> ")"                                    & expressão sequência     \\[.9em]

      <Exps>       &                                                     & sequência de expressões separadas por \verb|,| \\
      <Exps>       & <Exp> <ExpsRest>                                    &                         \\
      <ExpsRest>   &                                                     &                         \\
      <ExpsRest>   & "," <Exp> <ExpsRest>                                &                         \\[.9em]

      <ExpSeq>     &                                                     & sequência de expressões separadas por \verb|;| \\
      <ExpSeq>     & <Exp> <ExpSeqRest>                                  &                         \\
      <ExpSeqRest> &                                                     &                         \\
      <ExpSeqRest> & ";" <Exp> <ExpSeqRest>                              &                         \\[.9em]

      <Var>        & "id"                                                & variável simples        \\[.9em]

      <Decs>       & <Dec>                                               & sequência de declarações\\
      <Decs>       & <Dec> <Decs>                                        &                         \\[.9em]

      <Dec>        & <DecVar>                                            & declaração              \\[.9em]

      <DecVar>     & "var" "id" ":" "id" "=" <Exp>                       & declaração de variável  \\
      <DecVar>     & "var" "id" "=" <Exp>                                &                         \\[.9em]
    \end{longtable}
  \end{synshorts}

Observe que um \textbf{programa} em \lang{} é uma expressão.

A precedência relativa e a associatividade dos operadores é indicada
pela tabela a seguir, em ordem decrescente de precedência.
\begin{center}
  \begin{tabular}{|l|l|}\hline
    \textbf{operadores}                                                       & \textbf{associatividade} \\\hline
    \texttt{-} (unário)                                                       &                          \\\hline
    \texttt{\textasciicircum}                                                 & direita                  \\\hline
    \texttt{*}, \texttt{/}, \texttt{\%}                                       & esquerda                 \\\hline
    \texttt{+}, \texttt{-} (binário)                                          & esquerda                 \\\hline
    \texttt{=}, \texttt{<>}, \texttt{>}, \texttt{>=}, \texttt{<}, \texttt{<=} &                          \\\hline
    \texttt{\&\&}                                                             & esquerda                 \\\hline
    \texttt{||}                                                               & esquerda                 \\\hline
    \texttt{:=}                                                               &                          \\\hline
    \texttt{then}, \texttt{else}, \texttt{do}, \texttt{in}                    & direita                  \\\hline
  \end{tabular}
\end{center}


\end{document}
